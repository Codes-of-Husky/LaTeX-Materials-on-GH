%空行代表重启一个段落。
\chapter{Abstract}
%直接在奇数页页眉中显示章标题会多处一些章标题内部编号,这里重新定义\leftmark,后续所有章节都要重新定义
\renewcommand{\leftmark}{Abstract}
Eukaryotic cells evolved different types of organelles that carry out specialized functions. One of them is the cilium, also known as flagellum (interchangeable terms), which is found on most eukaryotic cells. Cilia are hairlike microtubule-based organelles that protrude from the cell surface and are composed of ciliary membrane, ciliary matrix and axoneme templated by a basal body. Cilia primarily play two vital roles. One is to function as a cellular motor to move either the cell itself or surrounding liquids/particles. The other function of cilia is to serve as a hub for cellular signaling. Defects in ciliary structures and functions in humans lead to diseases commonly referred to as `ciliopathies', such as blindness, deafness, hyperdactylism, heterotaxy and polycystic kidney disease.

The formation and maintenance of cilia, as well as ciliary signaling, depend on intraflagellar transport (IFT), a bidirectional movement of granular particles between the outer doublet microtubules and the flagellar membrane along the axoneme. More than 22 IFT proteins form at least three biochemically distinct complexes, namely IFT-A, IFT-B1 and IFT-B2. In addition to the dynamic movement of IFT in cilia, nearly all subunits of IFT complex, cargos and motor proteins are concentrated at the basal body where they upload, download or couple. However, the mechanism of the basal body localization of IFT subunits, which is one of the key steps in the initiation of ciliogenesis, remains largely unknown.

In this study, we explored the molecular mechanism of the basal body localization of IFT46. Firstly, we fused Citrine Yellow fluorescent protein (YFP) to the C-terminus of IFT46. When expressed in \textit{ift46-1}, rescued strains have full-length flagella and swim normally. The average ciliary lengths and the percentages of ciliated cells of the rescued strains are akin to those of wild-type (WT) cells. The anterograde and retrograde transport velocities and frequencies of IFT46::YFP are similar to results published previously. In summary, these data demonstrate that the 28 kDa YFP tag does not affect the function of IFT46. The confocal imaging results showed that the negative control, YFP, accumulates around the nuclei in \textit{ift46-1 YFP}. IFT46::YFP accumulates at basal bodies and localizes in a dotted pattern along the flagella, just as other IFT subunits do. Moreover, the punctate pattern of IFT46::YFP along the flagella is more obvious when examined by total internal reflection fluorescence microscopy.  To identify the targeting sequence in IFT46 responsible for its basal body localization, we generated and expressed a series of truncated IFT46 constructs fused with YFP in \textit{ift46-1}. We found that IFT46-C1 may contain the basal body targeting sequence of IFT46. Further research showed that the C-terminus without the glycine-rich tail (246-321 aa, BBTS3) could also target YFPs to the basal body. Moreover, we found that BBTS3 can also target IFT46 to cilia. IFT46-C1 and BBTS3 are able to move along the axoneme, and are
likely to be incorporated into IFT-B complexes as part of IFT46. These results demonstrate that the basal body and ciliary targeting of IFT46 require BBTS3.

We next determined whether the basal body localization of IFT46 depends on other IFT components. Accordingly, we expressed full-length IFT46::YFP and IFT46-C1::YFP in \textit{bld1}, \textit{ift88}, \textit{fla10-2} and \textit{dhc1b}. Positive transformants expressing IFT46::YFP or IFT46-C1::YFP were screened using western blotting with an antibody against GFP. We also examined the subcellular localization of IFT46 in \textit{ift81-2} and \textit{ift122-1} strains using immunostaining. The basal body localization of IFT46 is independent of IFT122, IFT88, IFT81, FLA10 or DHC1b. Surprisingly, the basal body localization of IFT46 depends upon IFT52, but not vice versa. IFT46::YFP and IFT46-C1::YFP were targeted to the basal body region in \textit{bld1 IFT46::YFP IFT52::3HA} and \textit{bld1 IFT46-C1::YFP IFT52::3HA}. These results further demonstrate that the basal body localization of IFT46 relies on IFT52. IFT46-C1 interacts with IFT52 through its C1 domain, which is mediated by the L285 and L286. That means IFT52 binds to and recruits IFT46 to the basal body region. Finally, we find that NLS-tagged IFT52C can recruit IFT46 to nuclei. Based on previous studies and our results here, we conclude that IFT52 and IFT46 are preassembled in the cytoplasm and are delivered to the basal body through vesicular transport or nonvesicle-mediated ways.

Our study here paves the way for the comprehensive deciphering of the mysteries of the basal body localizaiton of IFT proteins. We also shed new light on the atlas of interactions between the IFT proteins, motors and cargos at the basal body. These progresses are of vital importance to resolve the mechanism of ciliary assembly and disassembly. Furthermore, to some extent, our study can guide the prevention and treatment of ciliopathies through the regulation of ciliogenesis and ciliary functions.

\vspace{10mm}

\noindent \textbf{Key Words:\ }\textit{Chlamydomonas}, Intraflagellar transport, Basal body, Targeting sequence
