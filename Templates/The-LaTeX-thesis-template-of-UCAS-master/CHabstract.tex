%空行代表重启一个段落。
\chapter{摘要}
%直接在奇数页页眉中显示章标题会多处一些章标题内部编号,这里重新定义\leftmark,后续所有章节都要重新定义
\renewcommand{\leftmark}{摘要}
真核细胞利用不同的细胞器执行特定功能,其中之一是纤毛/鞭毛(可互换使用的术语)。纤毛是突出在绝大多数真核细胞表面的毛发状细胞器,由轴丝、基质和纤毛膜以基体为模板形成,可执行运动、感受和信号转导功能。其结构和功能异常可导致纤毛病,如眼盲、耳聋、多指、内脏异位和多囊肾等。

纤毛的组装、维持及参与信号转导功能依赖鞭毛内运输(intraflagellar transport, IFT)。IFT\ 是轴丝和纤毛膜之间的颗粒物沿轴丝作双向运动。IFT\ 复合物至少由\ 22\ 种蛋白组成。除定位在鞭毛内,IFT\ 蛋白还富集在基体周围参与货物的装卸及与分子马达的偶联。然而,作为鞭毛形成的关键初始步骤,IFT\ 蛋白基体定位的分子机制仍不明确。

本研究中我们对\ IFT\ 复合物\ B\ 中的一个亚基\ IFT46\ 的基体定位机制进行了研究。首先,我们使用\ C\ 端融合了黄色荧光蛋白的\ IFT46\ 恢复了\ \textit{IFT46}\ 的缺失突变体\ \textit{ift46-1}\ 的表型。鞭毛长度、鞭毛率、IFT\ 运动速率和频率的测定结果表明\ YFP\ 并没有影响\ IFT46\ 的功能。共聚焦和全内反射荧光显微成像结果显示阴性对照\ YFP\ 主要富集在细胞核周围,IFT46::YFP\ 聚集在基体周围且在鞭毛中呈点状分布。这与其他\ IFT\ 蛋白的定位特征是一致的。随后我们通过在\ \textit{ift46-1}\ 中表达截短的\ IFT46\ 鉴定了\ IFT46\ 可能的基体定位序列\ IFT46-C1 (246-344 aa)。进一步研究发现\ IFT46\ 的\ C\ 端(246-321 aa,BBTS3)也能够将黄色荧光蛋白靶定到基体。同时我们还发现\ BBTS3\ 也是\ IFT46\ 定位到纤毛所必须的。BBTS3::YFP\ 能够通过与其他\ IFT-B\ 亚基的相互作用组装到\ IFT\ 复合物中并沿鞭毛作双向运动。这些结果表明\ IFT46\ 的基体和纤毛定位需要\ BBTS3\ 的参与。

上述结果暗示\ IFT46\ 的基体和纤毛定位可能受\ IFT\ 相关蛋白的影响。为确定这些蛋白,我们在\ IFT\ 和分子马达的缺失突变体中表达\ IFT46::YFP\ 并筛选出高表达藻株。活细胞成像和免疫荧光结果表明\ IFT46\ 的基体定位依赖\ IFT52,但不依赖\ IFT122、IFT88、IFT81、FLA10\ 或\ DHC1b。在\ \textit{bld1 IFT46::YFP}\ 中表达\ IFT52::3HA\ 可恢复\ IFT46\ 的基体定位。然而,在\ \textit{ift46-1}\ 中\ IFT52::YFP\ 可定位在基体。这表明\ IFT52\ 的基体定位不依赖\ IFT46,IFT52\ 作用在\ IFT46\ 的上游。进一步研究发现\ IFT46\ 通过其\ C\ 端与\ IFT52\ 的\ C\ 端相互作用而影响其定位。这一相互作用主要由\ IFT46\ 第\ 285\ 位和\ 286\ 位的亮氨酸介导。这表明\ IFT52\ 可结合并招募\ IFT46。 最后,我们发现将\ IFT52\ 的\ C\ 端异位表达在衣藻细胞核可使\ IFT46\ 富集在细胞核。结合已有研究,我们的结果表明\ IFT52\ 和\ IFT46\ 可在胞质中预组装成亚复合物并通过囊泡介导或非囊泡介导的未知途径定位到基体。

我们的研究有助于最终阐明\ IFT\ 蛋白基体定位以及与货物和分子马达相互作用的机制。这对解析纤毛的组装与解聚机理至关重要。同时,这些研究也对调控纤毛形成和纤毛功能,预防和治疗纤毛病具有重要的指导意义。

\vspace{10mm}

\noindent \textbf{关键词:}衣藻,鞭毛内运输,基体,定位序列









