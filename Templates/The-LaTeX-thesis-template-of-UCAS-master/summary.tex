%空行代表重启一个段落。
%开始插入附录
%\appendix
\chapter{总\quad 结}
%直接在奇数页页眉中显示章标题会多处一些章标题内部编号,这里重新定义\leftmark,后续所有章节都要重新定义
\renewcommand{\leftmark}{总\quad 结}
纤毛/鞭毛是一种保守的细胞器,其结构和功能异常会导致纤毛病。纤毛的形成与维持依赖\ IFT,而\ IFT\ 蛋白聚集在基体是纤毛形成的关键初始步骤。本论文以衣藻\ IFT-B\ 复合物中的一个亚基\ IFT46\ 为对象对其基体定位的分子机制进行了研究,主要结果及结论如下:

一、利用\ IFT46::YFP\ 可恢复\ \textit{IFT46}\ 的缺失突变体\ \textit{ift46-1}\ 的表型。互补后的
藻株\textit{ift46-1 IFT46::YFP}\ 的鞭毛相关表型与野生型细胞无显著差异。这些结果表明融合黄色荧光蛋白\ YFP\ 不影响\ IFT46\ 的功能。

二、YFP\ 蛋白单独表达时主要定位在胞质,尤其是细胞核周围,核内\ YFP\ 蛋白含量较低。而\ IFT46::YFP\ 主要定位在基体,在鞭毛中呈点状分布且能双向运动。

三、通过在\ \textit{ift46-1}\ 和\ CC-125\ 中表达\ IFT46\ 的截短片段,我们发现\ IFT46-C1\ 和\ BBTS3\ 是\ IFT46\ 的基体和纤毛定位序列。

四、IFT46-C1\ 可与\ IFT-B\ 复合物中的其他亚基相互作用从而沿轴丝双向运动。

五、IFT46\ 的基体定位不依赖\ IFT122、IFT88、IFT81、FLA10\ 或\ DHC1b。但是其基体定位依赖\ IFT52。然而,IFT52\ 的基体定位并不依赖\ IFT46,这表明\ IFT52\ 作用在\ IFT46\ 的上游。通过生化和细胞生物学手段,我们发现\ IFT52\ 能够结合并招募\ IFT46\ 至基体。破坏\ IFT52\ 与\ IFT46\ 之间的相互作用可导致\ IFT46-C1\ 无法继续定位在基体。

六、带\ NLS\ 标签的\ IFT52C\ 可将\ IFT46\ 招募到细胞核,这表明\ IFT52\ 和\ IFT46\ 的预组装发生在胞质或高尔基体而非基体。

这些结果使我们对\ IFT46\ 基体定位的机制有了更深入的了解,为全面研究\ IFT\ 蛋白基体定位的机制及\ IFT\ 蛋白与分子马达和货物相互作用奠定了良好的基础,具有重要的理论意义。长远来看,IFT蛋白基体定位机制的解析有助于调控纤毛的组装与解聚,对纤毛病的预防和治疗具有积极作用。本研究的创新之处有以下三点:

首先,通过表达截短片段的方式首次鉴定了一个\ IFT\ 蛋白的基体/纤毛定位序列。再次,利用体内实验我们首次证明\ IFT52\ 和\ IFT46\ 之间的相互作用介导了\ IFT46\ 的基体定位。最后,利用体内方法初步证明\ IFT52\ 和\ IFT46\ 的预组装位置在细胞质而非基体,这一发现改变了人们关于\ IFT\ 蛋白在基体组装的传统观点。

这里我们鉴定了\ IFT46\ 的基体和纤毛定位序列,然而后续实验表明其上游运载蛋白是\ IFT52,这意味着我们鉴定到的并非经典的纤毛蛋白定位序列。后续研究应该从\ IFT52\ 或其他上游\ IFT\ 蛋白中的鉴定\ IFT\ 蛋白通用的定位序列及其相关运载因子。此外,IFT\ 蛋白在胞质尤其是高尔基体上预组装成亚复合物然后转运到基体可能是普遍存在的模式。在后续研究中应逐步理清\ IFT\ 蛋白的预组装网络并弄清鞭毛货物是否也参与预组装。同时,我们可以利用超高分辨显微镜或冷冻电镜研究\ IFT\ 蛋白在基体周围的精细定位和转位。最后,鞭毛是非常保守的细胞器,IFT\ 蛋白的基体和纤毛定位机制在不同类型的纤毛中很可能是通用的。尽管如此,在后续研究中我们还是应该同时在哺乳动物细胞初级纤毛中开展相关研究。
